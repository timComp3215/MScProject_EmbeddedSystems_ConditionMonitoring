\chapter{Conclusion and Future Works}

\section{Conclusion}




\section{Future Work}

This project is intended to provide a platform for future work on condition monitoring in the maritime industry.
As such, it is important to emphasise the areas that should be investigated.
This section lays out three areas for future research.

\subsubsection{FDD and the cloud}

Efforts should be made to implement more sophisticated FDD methods onto embedded systems for on-board processing.
As cloud computing becomes available on larger scales, the amount of data being generated for processing could become overwhelming \cite{FutureInternet}.
There is a need to perform processing at nodes and send limited amounts of data to the cloud \cite{CM_cloud}.
In the cloud, machine learning techniques can take specific statistical markers - as extacted by the embedded CMS - and train models to detect faults.
\par

Simple and useful improvements to the embedded CMS presented in this paper include:
\begin{itemize}
    \item Use in-place FFT to increase the number of sample which can be stored and therefore the resolution of the frequency domain
    \item Employ some form of envelope analysis to inspect different frequency ranges for statistical markers of condition
    \item Implement order tracking for vibration analysis
    \item Connect the embedded CMS to a cloud solution which performs analysis on the data
\end{itemize}

\subsubsection{Condition Monitoring Laboratory}

While the CML has provided a testbench for this project, distinct improvements are required to further develop and test CMSs.
Presently, it is simply too difficult to ensure a healthy condition will be maintained between tests.
Consequently, large amounts of time are spent calibrating the CML.
The separate elements of the CML should be fixed to prevent movement and misalignment.
\par

There is damage to the shaft, bearing and flexible coupling.
This, again, makes it hard to present a healthy condition against which other conditions can be compared.
These elements should be replaced and the method for swapping bearings, to use a faulty bearing for example, should be made easier.
\par

The system for inducing bending in the shaft needs to be improved.
Either recalibration should be possible or the linear actuator could be replaced with a stepper motor.
It is simply unreliable at inducing bending to a repeatable level.
\par

Acquiring a motor which has easier access to the internal components could allow for testing and diagnosis of electrical faults.
This would provide more information about the relative merits of MCSA and vibration analysis for a wider range of faults.
More complex MCSA diagnosis techniques such as Park's Vector could be assessed on the embedded system.

\subsubsection{Preparation for real world application}

Although the embedded CMS presented had some features which would be suitable in a real working environment, there are still significant changes which need to be made before it could be implemented on a ship.
Encasing of the processor board and sensors is essential to prevent damage to these components.
Suitable wiring should be used to prevent damage or interference in the signals.
\par

Before deployment of this system, it should be tested within a relatively controlled environment on board a ship.
Lloyd's Register have proposed testing on a local ferry.
This would be an excellent opportunity to find out how the system performs and what further challenges remain.
\par

Further work will be required to reduce the energy usage of the system and show that it can be powered in a sustainable manner.
By switching off sensors when they are not required and using low power modes for as long as possible, power consumption may be reduced to a level where the system can be powered from energy harvesting \cite{Embedded_Energy_Harvesting}.
This would have a great impact on CBM in the maritime industry, as the embedded CMS would become essentially free to run.
Even if the performance of the system is compromised slightly, this would present a much more realistic proposition for ship operators, builders and pump manufacturers alike \cite{CBM_norway_bd}.